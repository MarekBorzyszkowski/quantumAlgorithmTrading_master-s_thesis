\chapter{RYSUNKI, TABELE I INNE PODSTAWOWE KONSTRUKCJE}
\enlargethispage{20pt}
\section{Rysunki}
\label{rysunki}
Obrazki wstawiamy do dokumentu za pośrednictwem sekcji \textbf{figure}. Przykładowy wyśrodkowany rysunek z etykietą i opisem to Rys.~\ref{fig:obrazek}.

\begin{figure}[!h]
\centering
\includegraphics[width=0.7\textwidth]{images/obrazek.png}
\caption{Opis do obrazka.}
\label{fig:obrazek}
\end{figure}

W podanym przykładzie obrazek będzie wyśrodkowany, o szerokości odpowiadającej 70\% szerokości tekstu. Plik obrazka znajduje się w katalogu images i nazywa się obrazek.png. Latex dołoży starań, żeby umieścić go w tym miejscu (znacznik h w definicji figury). Wykrzyknik oznacza, że Latex bardzo się postara by obrazek tu był. Opisy obrazków zawsze znajdują się pod obrazkiem.

Latex potrafi wygenerować spis wszystkich numerowanych figur i umieścić go w miejscu, gdzie użyjemy polecenia \textbf{listoffigures}.

\subsection{Wielostronicowe obrazki}

Czasami zdarza się, że obrazek jest tak duży, że umieszczenie go na jednej stronie powoduje, że stanie się nieczytelny. Latex umożliwia pocięcie obrazka na kawałki, przeskalowanie i przetworzenie tych kawałków, oraz wyświetlenie ich w postaci następujących po sobie figur, gdzie tylko pierwsza wskazana będzie numerowana i uwzględniona w spisie rysunków. Przykład prezentuje Rys.~\ref{fig:duzy_obrazek}. Podpis pod każdą częścią będzie odmienny, numeracja będzie wspólna, w spisie wskazanie będzie do pierwszego z obrazków. Po szczegóły zapraszam do dokumentu w celu przeczytania komentarzy.


% definition of the scale of the picture
% here each part of the picture is rescaled
% so that it will fit page swidth
% 1255 is the original width of the picture
% later on we can use \ratio command as a macro
% that will rescale the fragment of the picture
\pgfmathsetmacro{\ratio}{\the\textwidth/1255}
% standard figure environment
\begin{figure}[h]
% centered
 \centering
% scaled by ratio fragment of the picture
% includegraphic*[vieport=lower_left_x lower_left_y upper_right_x upper_right_y]{image}
% in here we print upper part of the picture
   \scalebox{\ratio}{\includegraphics*[viewport=0 1951 1255 3795]{images/duzy_obrazek.png}}
% and the caption
   \caption{Bardzo duży obrazek}
\end{figure}

\begin{figure}[h]
% this tells Latex that this float is continuation of the previous one
% this way new number will not be assigned
\ContinuedFloat
% here we print lower part of the picture
 \centering
   \scalebox{\ratio}{\includegraphics*[viewport=0 0 1255 1951]{images/duzy_obrazek.png}}
 \caption[]{(ciąg dalszy)}
 \label{fig:duzy_obrazek}
\end{figure}

Analogicznie można również wstawiać i dzielić na kawałki obrazki zawarte w plikach pdf. Tutaj warto jednak rozmiary w mm podać, gdyż rozmiar po pikselach potrafi się rozjechać. Przykład efektu działania znajduje się na obrazku~\ref{fig:mw}. Przykład pochodzi od mgr inż.\ Michała Wójcika.

\begin{figure}[htb!]
  \centering
  \includegraphics*[viewport=0 153mm 175mm 306mm, scale=0.85]{images/producer-supervisor-perofmers.pdf}
  \caption{Duży obrazek w pdf}
\end{figure}

\begin{figure}[htb!]
  \ContinuedFloat
  \centering
  \includegraphics*[viewport=0 0 175mm 153mm, scale=0.85]{images/producer-supervisor-perofmers.pdf}
  \caption{(continued - UWAGA - brak [] po caption powoduje widoczność podpisu kontynuacji na wykazie!)}
  \label{fig:mw}
\end{figure}

Należy zauważyć, że Latex dołoży wszelkich starań, by obrazek umieścić tam gdzie go sobie zażyczyliśmy. Czasami jest to jednak niemożliwe, ze względu na zbyt małą ilość tekstu i niestety trzeba pokombinować. Typowe zabiegi to próba dodania znacznika [!h] przy definicji figury, wstawienie obrazka wcześniej w tekście czy też lekka zmiana jego rozmiaru.

\subsection{Wymuszanie renderowania obrazków do pewnego miejsca}

Czasami zdarza się, że Latex zbyt długo zwleka z renderowanie obrazków, przez co w pamięci podręcznej zaczyna brakować miejsca. Najczęściej problem ten występuje, gdy dokument zawiera bardzo dużo obrazków, a raczej niewiele tekstu, lub relacje pomiędzy rozmiarami obrazków a liczbą słów są zachwiane i nie jest możliwe ich poprawne rozlokowanie.

Wszystkie zaległe obrazki zawsze renderowane są na koniec rozdziału, jednak ich duże nagromadzenie może spowodować błąd przepełnienia pamięci. Czasami też koniec rozdziału jest zbyt odległym miejscem i lepiej po prostu wymusić renderowanie zaległych ramek w określonym miejscu. Służy temu komenda \textbf{$\backslash$clearpage}. Jej zastosowanie spowoduje renderowanie wszystkich, do tej pory nierozlokowanych obrazków.

\section{Listingi}
\label{listingi}
\enlargethispage{-5cm}
Tworzenie listingów umożliwia pakiet \textbf{listings}. Należy zwrócić uwagę, że etykiety i opisy listingów dodaje się w dośc specyficzny sposób jako parametry do bloku \textbf{lstlisting} (patrz kod dokumentu). Używając funkcji \textbf{lstset} możliwa jest zmiana kroju czcionki na inną niż reszty dokumentu. W przykładzie (Listing~\ref{kod:listingA}) zmniejszono czcionkę do rozmiaru indeksu dolnego.

\lstset{basicstyle=\scriptsize}
\begin{lstlisting}[frame=single,caption={Przykładowy listing},label=kod:listingA]
<?xml version="1.0"?>

<!DOCTYPE rdf:RDF [
 <!ENTITY owl "http://www.w3.org/2002/07/owl#" >
 <!ENTITY xsd "http://www.w3.org/2001/XMLSchema#" >
 <!ENTITY rdfs "http://www.w3.org/2000/01/rdf-schema#" >
 <!ENTITY rdf "http://www.w3.org/1999/02/22-rdf-syntax-ns#" >
]>

<rdf:RDF xmlns="http://kask.eti.pg.gda.pl/securityA.owl#"
 xml:base="http://kask.eti.pg.gda.pl/securityA.owl"
 xmlns:rdfs="http://www.w3.org/2000/01/rdf-schema#"
 xmlns:owl="http://www.w3.org/2002/07/owl#"
 xmlns:xsd="http://www.w3.org/2001/XMLSchema#"
 xmlns:rdf="http://www.w3.org/1999/02/22-rdf-syntax-ns#">
 <owl:Ontology rdf:about="http://kask.eti.pg.gda.pl/securityA.owl"/>

 <!-- 
 ///////////////////////////////////////////////////////////////////////////////////////
 //
 // Classes
 //
 ///////////////////////////////////////////////////////////////////////////////////////
 -->

 <!-- http://kask.eti.pg.gda.pl/securityA.owl#DSA -->

 <owl:Class rdf:about="http://kask.eti.pg.gda.pl/securityA.owl#DSA">
  <rdfs:subClassOf rdf:resource="http://kask.eti.pg.gda.pl/securityA.owl#asymetryczne"/>
 </owl:Class>

  
 <!-- http://kask.eti.pg.gda.pl/securityA.owl#RSA -->

 <owl:Class rdf:about="http://kask.eti.pg.gda.pl/securityA.owl#RSA">
  <rdfs:subClassOf rdf:resource="http://kask.eti.pg.gda.pl/securityA.owl#asymetryczne"/>
 </owl:Class>

  
 <!-- http://kask.eti.pg.gda.pl/securityA.owl#asymetryczne -->

 <owl:Class rdf:about="http://kask.eti.pg.gda.pl/securityA.owl#asymetryczne"/>
  

 <!-- http://kask.eti.pg.gda.pl/securityA.owl#symetryczne -->

 <owl:Class rdf:about="http://kask.eti.pg.gda.pl/securityA.owl#symetryczne"/>
</rdf:RDF>

\end{lstlisting}

\section{Algorytmy}
\label{algorytmy}
Pakiety \textbf{algorithmic} oraz \textbf{algorithm} dostarczają środowiska do definiowania algorytmów. Można za ich pomocą definiować zarówno pseudokod jak i algorytmy ogólne. Pseudokod przedstawia Algorytm~\ref{tboi:alg}. Przykładową ogólną reprezentację algorytmu Levenshteina obrazuje Algorytm~\ref{tboi:alg_leven}.

\begin{algorithm}
 \floatname{algorithm}{Algorytm}
 \caption{Pseudokod prezentujący jakiś bezsensowny algorytm}
 \label{tboi:alg}
 \begin{algorithmic}[1]
\STATE \textbf{program} FancyProgram \{

\STATE \textbf{function} superFunkcja(List listA) \{

\FORALL{element :~listA.getElements()}
\STATE int result = storeElement(element);
\IF{result == SUCCESS}
\STATE double number = result*element;
\ELSIF{result == FAILURE}
\STATE double number = result/element;
\ENDIF
\ENDFOR
\STATE \}

\STATE input List elements;
\STATE output number;

\STATE superFunkcja(elements);
\RETURN number;
\STATE \}
 \end{algorithmic}
\end{algorithm}

\begin{algorithm}
 \floatname{algorithm}{Algorytm}
 \caption{Algorytm Levenshteina}
 \label{tboi:alg_leven}
%  \begin{algorithmic}[1]
\begin{enumerate}
 \item Niech $n = \textit{długość}(s)$ a~$m = \textit{długość}(t)$.
 \item Jeżeli $n == 0$ zwróć m~i zakończ działanie.
 \item Jeżeli $m == 0$ zwróć n~i zakończ działanie.
 \item Utwórz macierz $d$ o~$m+1$ wierszach indeksowanych od $0$ do $m$ oraz $n+1$ kolumnach indeksowanych od $0$ do $n$, pierwszy wiersz zainicjuj wartościami od $0$ do $n$ a~pierwszą kolumnę od $0$ do $m$.
 \item Porównaj każdy znak pierwszego ciągu, indeksowany za pomocą $i$, z~każdym znakiem drugiego ciągu, indeksowanym za pomocą $j$. Dla każdego porównania:
\begin{enumerate}
 \item jeżeli $s[i]$ jest identyczne z~$t[j]$, $koszt = 0$, w~przeciwnym wypadku $koszt = 1$,
 \item ustaw wartość komórki $d[i,j]$ macierzy na wartość minimalną spośród:
\begin{itemize}
 \item $d[i-1,j] +~1$,
 \item $d[i,j-1] +~1$,
 \item $d[i-1,j-1] +~koszt$.
\end{itemize}
\end{enumerate}
 \item Odczytaj odległość $dist_{lev}(s,t)$ z~komórki $d[n,m]$.
\end{enumerate}
%\end{algorithmic}
\end{algorithm}

Podobnie jak w przypadku rysunków, algorytmy są automatycznie rozlokowywane w tekście przez system Latex.

\section{Tabele}

Latex pozwala na konstruowanie bardzo rozbudowanych tabel. Pełny opis możliwości znaleźć można w sieci Internet, a samo zagadnienie jest bardzo skomplikowane. Prosty przykład prezentuje Tabela~\ref{tab:tab1}.

\begin{table}[h]
\caption{Opis tabelki}
\label{tab:tab1}
 \centering
\begin{tabular}{|c|p{4.5cm}|p{6cm}|}
  \hline
  Pole 1 & Pole 2 & Pole 3\\
  \hline
  1 & Pojęcie & Opis tegoż pojęcia\\
  \hline
  2 & Pojęcie & Opis tegoż pojęcia\\
  \hline
\end{tabular}
% \vspace{-0.5cm}
\end{table}

Pakiet \textbf{longtable} pozwala na generowanie ogromnych tabel rozłożonych na kilka stron.

% this changes spacing between table rows, default is one, needs to be reset after the table!
\renewcommand{\arraystretch}{0.85}
% longtable definition, headers are defined in the same manner as normal table
% here we change fontsize in each collumn to size smaller than tine defined in the preamble
% all keywords within the table definition are the same as in standard table
\begin{longtable}[h!]{|>{\tinyv}l|>{\tinyv}c|>{\tinyv}c|>{\tinyv}c|>{\tinyv}c|>{\tinyv}c|>{\tinyv}c|>{\tinyv}c|>{\tinyv}c|>{\tinyv}c|>{\tinyv}c|}
\caption{Bardzo długa tabela (główna etykieta)\label{tab:long_table}}\\
\endfirsthead
\caption[]{(ciąg dalszy -- ten tekst pokazywać się będzie na kolejnych stronach)}\\
\endhead
\hline
Lemma B~& \multicolumn{5}{>{\tinyv}c|}{Agency} & \multicolumn{5}{>{\tinyv}c|}{Area} \\
\hline
Lemma A~& $max(P_{lex},P_{sem})$ & $P_{kom}$ & $P_{str}$ & $P_{sk}$ & Wynik & $max(P_{lex},P_{sem})$ & $P_{kom}$ & $P_{str}$ & $P_{sk}$ & Wynik \\
\hline
Asset & 0,30 & 0,00 & 0,06 & 0,02 & Różne & 0,40 & 0,00 & 0,06 & 0,02 & Różne \\
\hline
Attack & 0,33 & 0,00 & 0,09 & 0,03 & Różne & 0,33 & 0,00 & 0,09 & 0,03 & Różne \\
\hline
Circumstance & 0,37 & 0,00 & 0,00 & 0,00 & Różne & 0,85 & X~& X~& X~& Rodzeństwo \\
\hline
Continuity & 0,31 & 0,00 & 0,13 & 0,04 & Różne & 0,42 & 0,00 & 0,13 & 0,04 & Różne \\
\hline
Error & 0,32 & 0,00 & 0,08 & 0,03 & Różne & 0,35 & 0,00 & 0,08 & 0,03 & Różne \\
\hline
Event & 0,46 & 0,00 & 0,10 & 0,03 & Różne & 0,40 & 0,00 & 0,10 & 0,03 & Różne \\
\hline
Group & 0,60 & 0,00 & 0,00 & 0,00 & Różne & 0,55 & 0,00 & 0,00 & 0,00 & Różne \\
\hline
Harm & 0,37 & 0,00 & 0,00 & 0,00 & Różne & 0,42 & 0,00 & 0,00 & 0,00 & Różne \\
\hline
Mission & 0,46 & 0,00 & 0,10 & 0,03 & Różne & 0,19 & 0,00 & 0,10 & 0,03 & Różne \\
\hline
Operation & 0,83 & X~& X~& X~& Rodzeństwo & 0,43 & 0,00 & 0,50 & 0,15 & Różne \\
\hline
Organization & 0,72 & X~& X~& X~& B~podrzędne A~& 0,35 & 0,00 & 0,10 & 0,03 & Różne \\
\hline
Potential & 0,36 & 0,00 & 0,00 & 0,00 & Różne & 0,41 & 0,00 & 0,00 & 0,00 & Różne \\
\hline
Risk & 0,32 & 0,00 & 0,17 & 0,05 & Różne & 0,20 & 0,00 & 0,17 & 0,05 & Różne \\
\hline
Security & 0,65 & 0,00 & 0,00 & 0,00 & Różne & 0,40 & 0,00 & 0,00 & 0,00 & Różne \\
\hline
Threat & 0,28 & 0,00 & 0,00 & 0,00 & Różne & 0,50 & 0,00 & 0,00 & 0,00 & Różne \\
\hline
Value & 0,30 & 0,00 & 0,11 & 0,03 & Różne & 0,47 & 0,00 & 0,11 & 0,03 & Różne \\
\hline
Vulnerability & 0,31 & 0,00 & 0,00 & 0,00 & Różne & 0,42 & 0,00 & 0,00 & 0,00 & Różne \\
\hline
Weakness & 0,34 & 0,00 & 0,11 & 0,03 & Różne & 0,44 & 0,00 & 0,11 & 0,03 & Różne \\
\hline \hline
Lemma B~& \multicolumn{5}{>{\tinyv}c|}{Asset} & \multicolumn{5}{>{\tinyv}c|}{Attack} \\
\hline
Lemma A~& $max(P_{lex},P_{sem})$ & $P_{kom}$ & $P_{str}$ & $P_{sk}$ & Wynik & $max(P_{lex},P_{sem})$ & $P_{kom}$ & $P_{str}$ & $P_{sk}$ & Wynik \\
\hline
Asset & 1,00 & X~& X~& X~& Tożsame & 0,17 & 0,00 & 0,39 & 0,12 & Różne \\
\hline
Attack & 0,17 & 0,00 & 0,85 & 0,25 & Różne & 1,00 & X~& X~& X~& Tożsame \\
\hline
Circumstance & 0,32 & 0,34 & 0,00 & 0,24 & Różne & 0,25 & 0,00 & 0,00 & 0,00 & Różne \\
\hline
Continuity & 0,27 & 0,00 & 0,50 & 0,15 & Różne & 0,26 & 0,00 & 0,25 & 0,08 & Różne \\
\hline
Error & 0,48 & 0,25 & 0,47 & 0,31 & Różne & 0,45 & 0,00 & 0,27 & 0,08 & Różne \\
\hline
Event & 0,32 & 0,20 & 0,64 & 0,33 & Różne & 0,51 & 0,00 & 0,42 & 0,13 & Różne \\
\hline
Group & 0,13 & 0,00 & 0,00 & 0,00 & Różne & 0,31 & 0,00 & 0,00 & 0,00 & Różne \\
\hline
Harm & 0,33 & 0,26 & 0,00 & 0,18 & Różne & 0,51 & 0,00 & 0,00 & 0,00 & Różne \\
\hline
Mission & 0,29 & 0,00 & 0,44 & 0,13 & Różne & 0,77 & X~& X~& X~& Rodzeństwo \\
\hline
Operation & 0,31 & 0,00 & 0,15 & 0,05 & Różne & 0,95 & X~& X~& X~& B~podrzędne A~\\
\hline
Organization & 0,55 & 0,31 & 0,35 & 0,33 & Różne & 0,76 & X~& X~& X~& Rodzeństwo \\
\hline
Potential & 0,32 & 0,22 & 0,00 & 0,16 & Różne & 0,46 & 0,00 & 0,00 & 0,00 & Różne \\
\hline
Risk & 0,20 & 0,29 & 0,46 & 0,34 & Różne & 0,40 & 0,00 & 0,18 & 0,05 & Różne \\
\hline
Security & 0,31 & 0,00 & 0,00 & 0,00 & Różne & 0,39 & 0,00 & 0,00 & 0,00 & Różne \\
\hline
Threat & 0,33 & 0,23 & 0,00 & 0,16 & Różne & 0,40 & 0,00 & 0,00 & 0,00 & Różne \\
\hline
Value & 0,64 & 0,00 & 0,38 & 0,11 & Różne & 0,20 & 0,00 & 0,14 & 0,04 & Różne \\
\hline
Vulnerability & 0,27 & 0,31 & 0,00 & 0,22 & Różne & 0,08 & 0,00 & 0,00 & 0,00 & Różne \\
\hline
Weakness & 0,47 & 0,22 & 0,38 & 0,27 & Różne & 0,13 & 0,00 & 0,14 & 0,04 & Różne \\
\hline \hline
Lemma B~& \multicolumn{5}{>{\tinyv}c|}{Circumstance} & \multicolumn{5}{>{\tinyv}c|}{Countermeasure} \\
\hline
Lemma A~& $max(P_{lex},P_{sem})$ & $P_{kom}$ & $P_{str}$ & $P_{sk}$ & Wynik & $max(P_{lex},P_{sem})$ & $P_{kom}$ & $P_{str}$ & $P_{sk}$ & Wynik \\
\hline
Asset & 0,32 & 0,00 & 0,11 & 0,03 & Różne & 0,14 & 0,00 & 0,00 & 0,00 & Różne \\
\hline
Attack & 0,25 & 0,00 & 0,18 & 0,05 & Różne & 0,31 & 0,00 & 0,00 & 0,00 & Różne \\
\hline
Circumstance & 1,00 & X~& X~& X~& Tożsame & 0,21 & 0,00 & 0,00 & 0,00 & Różne \\
\hline
Continuity & 0,33 & 0,00 & 0,25 & 0,08 & Różne & 0,29 & 0,00 & 0,00 & 0,00 & Różne \\
\hline
Error & 0,27 & 0,00 & 0,17 & 0,05 & Różne & 0,21 & 0,00 & 0,00 & 0,00 & Różne \\
\hline
Event & 0,99 & X~& X~& X~& A~podrzędne B~& 0,26 & 0,00 & 0,00 & 0,00 & Różne \\
\hline
Group & 0,17 & 0,00 & 0,00 & 0,00 & Różne & 0,14 & 0,00 & 0,00 & 0,00 & Różne \\
\hline
Harm & 0,52 & 0,00 & 0,00 & 0,00 & Różne & 0,33 & 0,00 & 0,00 & 0,00 & Różne \\
\hline
Mission & 0,25 & 0,00 & 0,20 & 0,06 & Różne & 0,20 & 0,00 & 0,00 & 0,00 & Różne \\
\hline
Operation & 0,41 & 0,00 & 0,33 & 0,10 & Różne & 0,29 & 0,00 & 0,00 & 0,00 & Różne \\
\hline
Organization & 0,34 & 0,00 & 0,20 & 0,06 & Różne & 0,34 & 0,00 & 0,00 & 0,00 & Różne \\
\hline
Potential & 0,39 & 0,00 & 0,00 & 0,00 & Różne & 0,29 & 0,00 & 0,00 & 0,00 & Różne \\
\hline
Risk & 0,23 & 0,00 & 0,33 & 0,10 & Różne & 0,21 & 0,00 & 0,00 & 0,00 & Różne \\
\hline
Security & 0,49 & 0,00 & 0,00 & 0,00 & Różne & 0,73 & X~& X~& X~& Rodzeństwo \\
\hline
Threat & 0,21 & 0,00 & 0,00 & 0,00 & Różne & 0,29 & 0,00 & 0,00 & 0,00 & Różne \\
\hline
Value & 0,33 & 0,00 & 0,22 & 0,07 & Różne & 0,21 & 0,00 & 0,00 & 0,00 & Różne \\
\hline
Vulnerability & 0,43 & 0,00 & 0,00 & 0,00 & Różne & 0,21 & 0,00 & 0,00 & 0,00 & Różne \\
\hline
Weakness & 0,36 & 0,00 & 0,22 & 0,07 & Różne & 0,14 & 0,00 & 0,00 & 0,00 & Różne \\
\hline \hline
Lemma B~& \multicolumn{5}{>{\tinyv}c|}{Device} & \multicolumn{5}{>{\tinyv}c|}{Event} \\
\hline
Lemma A~& $max(P_{lex},P_{sem})$ & $P_{kom}$ & $P_{str}$ & $P_{sk}$ & Wynik & $max(P_{lex},P_{sem})$ & $P_{kom}$ & $P_{str}$ & $P_{sk}$ & Wynik \\
\hline
Asset & 0,08 & 0,00 & 0,00 & 0,00 & Różne & 0,32 & 0,13 & 0,39 & 0,21 & Różne \\
\hline
Attack & 0,37 & 0,00 & 0,00 & 0,00 & Różne & 0,51 & 0,00 & 0,64 & 0,19 & Różne \\
\hline
Circumstance & 0,21 & 0,00 & 0,00 & 0,00 & Różne & 0,99 & X~& X~& X~& B~podrzędne A~\\
\hline
Continuity & 0,27 & 0,00 & 0,00 & 0,00 & Różne & 0,32 & 0,00 & 0,25 & 0,08 & Różne \\
\hline
Error & 0,30 & 0,00 & 0,00 & 0,00 & Różne & 0,42 & 0,17 & 0,27 & 0,20 & Różne \\
\hline
Event & 0,34 & 0,00 & 0,00 & 0,00 & Różne & 1,00 & X~& X~& X~& Tożsame \\
\hline
Group & 0,13 & 0,00 & 0,00 & 0,00 & Różne & 0,24 & 0,00 & 0,00 & 0,00 & Różne \\
\hline
Harm & 0,40 & 0,00 & 0,00 & 0,00 & Różne & 0,51 & 0,18 & 0,00 & 0,12 & Różne \\
\hline
Mission & 0,24 & 0,00 & 0,00 & 0,00 & Różne & 0,37 & 0,00 & 0,21 & 0,06 & Różne \\
\hline
Operation & 0,27 & 0,00 & 0,00 & 0,00 & Różne & 0,57 & 0,00 & 0,29 & 0,09 & Różne \\
\hline
Organization & 0,41 & 0,00 & 0,00 & 0,00 & Różne & 0,41 & 0,22 & 0,13 & 0,20 & Różne \\
\hline
Potential & 0,22 & 0,00 & 0,00 & 0,00 & Różne & 0,67 & 0,14 & 0,00 & 0,10 & Różne \\
\hline
Risk & 0,26 & 0,00 & 0,00 & 0,00 & Różne & 0,41 & 0,18 & 0,18 & 0,18 & Różne \\
\hline
Security & 0,83 & X~& X~& X~& Rodzeństwo & 0,48 & 0,00 & 0,00 & 0,00 & Różne \\
\hline
Threat & 0,30 & 0,00 & 0,00 & 0,00 & Różne & 0,35 & 0,19 & 0,00 & 0,13 & Różne \\
\hline
Value & 0,18 & 0,00 & 0,00 & 0,00 & Różne & 0,32 & 0,00 & 0,14 & 0,04 & Różne \\
\hline
Vulnerability & 0,15 & 0,00 & 0,00 & 0,00 & Różne & 0,42 & 0,18 & 0,00 & 0,12 & Różne \\
\hline
Weakness & 0,25 & 0,00 & 0,00 & 0,00 & Różne & 0,35 & 0,23 & 0,14 & 0,21 & Różne \\
\hline \hline
Lemma B~& \multicolumn{5}{>{\tinyv}c|}{Group} & \multicolumn{5}{>{\tinyv}c|}{Harm} \\
\hline
Lemma A~& $max(P_{lex},P_{sem})$ & $P_{kom}$ & $P_{str}$ & $P_{sk}$ & Wynik & $max(P_{lex},P_{sem})$ & $P_{kom}$ & $P_{str}$ & $P_{sk}$ & Wynik \\
\hline
Asset & 0,13 & 0,00 & 0,00 & 0,00 & Różne & 0,33 & 0,00 & 0,00 & 0,00 & Różne \\
\hline
Attack & 0,31 & 0,00 & 0,00 & 0,00 & Różne & 0,51 & 0,00 & 0,00 & 0,00 & Różne \\
\hline
Circumstance & 0,17 & 0,00 & 0,00 & 0,00 & Różne & 0,52 & 0,00 & 0,00 & 0,00 & Różne \\
\hline
Continuity & 0,10 & 0,00 & 0,00 & 0,00 & Różne & 0,33 & 0,00 & 0,00 & 0,00 & Różne \\
\hline
Error & 0,20 & 0,00 & 0,00 & 0,00 & Różne & 0,45 & 0,00 & 0,00 & 0,00 & Różne \\
\hline
Event & 0,24 & 0,00 & 0,00 & 0,00 & Różne & 0,51 & 0,00 & 0,00 & 0,00 & Różne \\
\hline
Group & 1,00 & X~& X~& X~& Tożsame & 0,11 & 0,00 & 0,00 & 0,00 & Różne \\
\hline
Harm & 0,11 & 0,00 & 0,00 & 0,00 & Różne & 1,00 & X~& X~& X~& Tożsame \\
\hline
Mission & 0,46 & 0,00 & 0,00 & 0,00 & Różne & 0,26 & 0,00 & 0,00 & 0,00 & Różne \\
\hline
Operation & 0,17 & 0,00 & 0,00 & 0,00 & Różne & 0,39 & 0,00 & 0,00 & 0,00 & Różne \\
\hline
Organization & 0,87 & X~& X~& X~& A~podrzędne B~& 0,48 & 0,00 & 0,00 & 0,00 & Różne \\
\hline
Potential & 0,11 & 0,00 & 0,00 & 0,00 & Różne & 0,40 & 0,00 & 0,00 & 0,00 & Różne \\
\hline
Risk & 0,32 & 0,00 & 0,00 & 0,00 & Różne & 0,28 & 0,00 & 0,00 & 0,00 & Różne \\
\hline
Security & 0,43 & 0,00 & 0,00 & 0,00 & Różne & 0,50 & 0,00 & 0,00 & 0,00 & Różne \\
\hline
Threat & 0,17 & 0,00 & 0,00 & 0,00 & Różne & 0,25 & 0,00 & 0,00 & 0,00 & Różne \\
\hline
Value & 0,46 & 0,00 & 0,00 & 0,00 & Różne & 0,33 & 0,00 & 0,00 & 0,00 & Różne \\
\hline
Vulnerability & 0,09 & 0,00 & 0,00 & 0,00 & Różne & 0,44 & 0,00 & 0,00 & 0,00 & Różne \\
\hline
Weakness & 0,10 & 0,00 & 0,00 & 0,00 & Różne & 0,36 & 0,00 & 0,00 & 0,00 & Różne \\
\hline
\end{longtable}
% remember to reset table spacing!!!
\renewcommand{\arraystretch}{1}
